\chapter{Literature Survey}

One of the matured innovation in the field of electronics was the development of Wireless Sensor Network (WSN ) which paved the way for various research works. The miniaturization of the components has allowed the user to experiment in various field of science such as health care, military applications, traffic control, monitoring, data collection \cite{Khedo2017} \cite{Liu2017}. Out of these urban air quality monitoring has gained a lot of attention as it is one of the major issues faced by the society. With the help of low cost and energy efficient sensor network there are various research work done. This chapter explains the different methods of understanding and detecting air pollution which is classified based on the platform which is used for the measurement\cite{Yi2015} \cite{Pavani2017}.
\begin{enumerate}
    \item Vehicle based sensor network
    \item Wearable sensor network
    \item Community sensor network
    \item Static sensor network
 \end{enumerate} 
 The work done in these different categories will be explained in detail further in this chapter.

\section{Vehicle based sensor network}

The compact and low cost sensors are installed and attached to a vehicle like bus or cars in order to achieve a spatially resolved data. There has been an increasing number of mobile vehicles in the urban areas. This was taken as a medium for obtaining air pollution data from the environment in cities. Vehicular wireless sensor network \cite{Hu2009} proposed a system to study \textit{micro-climate monitoring} which means the changes in concentration of a single pollutant measurement ($CO2$ in this case) by mounting the sensor node onto a vehicle. The system is equipped with $CO2$ sensor, Global System for Mobile(GSM) module, GPS receiver and  an intra-vehicular network by ZigBee module. The collected data was transferred through GSM short messages to the server and is displayed on google maps for results. Later this same work was further improved \cite{Hu2011} by creating a simulation model and making changes to the existing prototype of the system. The working prototype is now divided into two which consists of a central unit which is inside the vehicle and an external unit to collect the environmental data. The external units sends the data to central unit which reports the data. The work also addressed two network related problem of reducing reporting overhead and reducing communication overhead on the cellular network as sensing short messages incurred charge. 

In Mobile Air Quality Monitoring Network (MAQUMON) \cite{Volgyesi2008} proposed a prototype in which the sensor node  measures $O3$, $CO$, and $NO2$ concentration are mounted on cars. The sensor node is connected to a microcontroller and is powered by a Li-ion battery whose life is limited for few hours. It is equipped with a GPS module for determining the location and the collected data is transferred to a laptop through a bluetooth module. When the system is in coverage of a Wi-Fi network the collected data is transferred to web server and visualized in sensor map web application. Another approach for Vehicular based mobile architecture is represented in \cite{Devarakonda2013}, which is a fine grained approach was provided to get spatially dense data of pollutants. It propose two data framing models one for a public transportation infrastructure and other for a personal sensing device. In public transportation a mobile sensing box (MSB) was installed in buses which contained the microcontroller, sensors for measuring $PM$ AND $ CO$, a GPS module and a cellular modem. The module was powered through bus batteries and the collected data were transferred to server and visualized using google fusion tables interfaces. In the second framing, the system was installed in a car and connected over a bluetooth to a smartphone . This gave the user aware about the air quality while driving through a specific area.
The wireless sensor deployment in bus service \cite{Saha2017} proposed the idea of placing the sensor nodes in public transport buses in order to obtain real time data of pollutants. The WSN here is divided into sensor node and sink node in which the former collects the data from environment and latter aggregates the data from the sensor node and transmits to a server via a long range radio bands. These sink nodes are either installed in a T-junctions or an X-junction where most of the buses cross.
Another paper that  discusses an effective method to acquire air quality data in urban area is \cite {Shirai2016} in which the proposed system has the sensing unit kept on a public vehicle which keeps on moving in and around the city. The system has made use of garbage trucks and garbage patrol vehicles. The main focus is on pollutants like $PM$, $CO$ and $SO2$ and is also aided with GPS module. To remotely monitor the sensing conditions and to check for maintenance, a control center tool has also been developed. It consists of a map, graph which tracks the route of the vehicles and the sensor data acquired by each vehicle. There was a  monitor which was developed along with the system to send the users the collected data. It estimates the amount of pollution inhaled by the user, by the acquiring the user's location from the mobile application and mapping it to the location-sensors value which has been already computed. It also considers the respiratory volume of the user to estimate the pollution inhalation.

Another method of transportation which was taken for understanding the air quality was public bicycle system. The bicycle borne sensor \cite{Xiang2016,Liu2015a} deploys an exhaust gas sensor and $PM$ sensor, GPS module and a microprocessor. The system collects data along with the location from the GPS module and is stored in micro SD card. The system is powered with the help pf Lithium polymer batteries. When the subscriber returns the bicycle to dock station the data is transferred to data centre via bluetooth module and is then visualized using 'baidu' heat map. To understand air quality by the use of Unmanned Aerial Vehicle (UAV) \cite{Zhi2017} as a medium is a different approach. In this the sensor board which measures six different pollutants which contribute to AQI index is mounted on the flight.  These flights are carried out in 30 days with 20 minutes of monitoring. The UAV also contains a smart phone. TRYING TO SEE




\section{Wearable sensor network}

\section{Community sensor sensor}

\section{Static sensor network}