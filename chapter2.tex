\chapter{Literature Survey}

One of the matured innovation in the field of electronics was the development of Wireless Sensor Network (WSN ) which paved the way for various research works. The miniaturization of the components has allowed the user to experiment in various field of science such as health care, military applications, traffic control, monitoring, data collection \cite{Khedo2017} \cite{Liu2017}. Out of these urban air quality monitoring has gained a lot of attention as it is one of the major issues faced by the society. With the help of low cost and energy efficient sensor network there are various research work done. This chapter explains the different methods of understanding and detecting air pollution which is classified based on the platform which is used for the measurement\cite{Yi2015} \cite{Pavani2017}.
\begin{enumerate}
    \item Vehicle based sensor network
    \item Wearable sensor network
    \item Community sensor network
    \item Static sensor network
 \end{enumerate} 
 The work done in these different categories will be explained in detail further in this chapter.

\section{Vehicle based sensor network}

The compact and low cost sensors are installed and attached to a vehicle like bus or cars in order to achieve a spatially resolved data. There has been an increasing number of mobile vehicles in the urban areas. This was taken as an advantage for environmental data collection in cities. Vehicular wireless sensor network \cite{Hu2009} proposed a system to study \textit{micro-climate monitoring} which means the changes in concentration of a single pollutant measurement ($CO2$ in this case) by mounting the sensor node onto a vehicle. The system is equipped with $CO2$ sensor, Global System for Mobile(GSM) module, GPS receiver and  an intra-vehicular network by ZigBee module. The collected data was transferred through GSM short messages to the server and is displayed on google maps for results. Later this same work was further improved \cite{Hu2011} by creating a simulation model and making changes to the existing prototype of the system. The working prototype is now divided into two that consists of a central unit which is inside the vehicle and an external unit to collect the environmental data. The external units sends the data to central unit which reports the data. The work also addressed two network related problem of reducing reporting overhead and reducing communication overhead on the cellular network as sensing short messages incurred charge.



\section{Wearable sensor network}

\section{Community sensor sensor}

\section{Static sensor network}