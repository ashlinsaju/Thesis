\chapter{Literature Survey}

One of the matured innovation in the field of electronics was the development of Wireless Sensor Network (WSN ) which paved the way for various research works. The miniaturization of the components has allowed the user to experiment in various field of science such as health care, military applications, traffic control, monitoring, data collection \cite{Khedo2017} \cite{Liu2017}. Out of these urban air quality monitoring has gained a lot of attention as it is one of the major issues faced by the society. With the help of low cost and energy efficient sensor network there are various research work done. This chapter explains the different methods of understanding and detecting air pollution which is classified based on the platform which is used for the measurement\cite{Yi2015} \cite{Pavani2017}.
\begin{enumerate}
    \item Vehicle based sensor network
    \item Wearable sensor network
    \item Community sensor network
    \item Static sensor network
 \end{enumerate} 
 The work done in these different categories will be explained in detail further in this chapter.

\section{Vehicle based sensor network}
The compact and low cost sensors are installed and attached to a vehicle like bus or cars in order to achieve a spatially resolved data.  For example in \cite{Hu2011} studies the change in concentration of a single pollutant ($CO2$) measurement which is termed as \textit{micro-climate monitoring} by deploying fewer sensor nodes on taxis or buses in city areas.
\section{Wearable sensor network}
\section{Community sensor sensor}
\section{Static sensor network}