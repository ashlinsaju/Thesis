\chapter{Literature Survey}

Development of Wireless Sensor Network (WSN) is considered as one of the considerable innovation in the field of electronics. The miniaturization of the components has allowed the user to explore more in various field of science such as health care, military applications, traffic control, monitoring, data collection \cite{Khedo2017} \cite{Liu2017}. Out of all the applications, urban air quality monitoring has gained a lot of attention as it is one of the major issues faced by the society today. There are different approaches for understanding air quality and this chapter gives an overview on the research work done in academic literature that is classified based on the platform used for the measurement\cite{Yi2015} \cite{Pavani2017}.
 
\begin{enumerate}

    \item Vehicle based sensor network
    \item Wearable sensor network
    \item Community sensor network
    \item Static sensor network

 \end{enumerate} 

 The research done in different platform will be explained in detail further in this chapter.

\section{Vehicle based sensor network (VSN)}

In recent times, the number of private vehicles on the road have increased in proportion to the increasing population \cite{Downs2004}. Even though the increase in automobile is one of the major factor that is contributing to the increase in pollution, certain researchers took this as a way for collecting air pollution data. In this category of work, the vehicles (like buses or cars) are installed with a portable, low cost sensor so as to obtain spatially resolved data. 
\par
%Portable and low cost sensors are installed and attached to a vehicle like buses or cars in order to achieve a spatially resolved data. There have been an increasing number of mobile vehicles in the urban areas. This was taken as a medium for obtaining air pollution data from the environment in cities. 

 One of the best way to study the quality of air is to obtain fine-grained data also called 'micro-climate monitoring'. However, with the existing monitoring system being bulky and expensive it is impossible to obtain spatially resolved data. In order to solve this issue, in 2009 a group of researchers used mobility as a method and proposed a Vehicular wireless sensor network \cite{Hu2009} that measures the changes in concentration of a single pollutant measurement (Carbon Dioxide in this case) by mounting the sensor node onto a vehicle. The system is equipped with Carbon Dioxide ($CO_2$) sensor, Global System for Mobile (GSM) module, GPS receiver and  an intra-vehicular network by ZigBee module. The collected data is transferred through GSM short messages to the server and is displayed on google maps for results. This work had two network related drawback one to reduce reporting overhead so as to avoid possible duplication of air pollution data and second is to reduce communication overhead on the cellular network as sensing short messages incurred charge which was later addressed in 2011 in \cite{Hu2011} by designing two algorithms. It also created a simulation model to verify the performance of algorithm. The research work mainly focused on data collection and optimization of the collected data. The problems like calibration of sensors, visualization of data and management of the VSN was not taken care of.

Another work in 2008 initiated by different group of researchers proposed Mobile Air Quality Monitoring Network (MAQUMON) \cite{Volgyesi2008} and is supported by microsoft proposed a prototype in which the sensor node  measuring Ozone ($O_3$), Carbon Monoxide ($CO$), and Nitrogen Dioxide ($NO_2$) concentration were mounted on cars for monitoring a large area. The sensor node is connected to a microcontroller and is powered by a Li-ion battery whose life is limited for few hours. It is equipped with a GPS module for determining the location and the collected data is transferred to a laptop through a bluetooth module. When the system is in coverage of a Wi-Fi network the collected data is transferred to web server and visualized in sensor map web application. The issue with this system is that it cannot transfer the data immediately for visualization. Another approach for Vehicular based mobile architecture is represented in \cite{Devarakonda2013}, which uses a fine grained approach was provided to get spatially dense data of pollutants. It propose two data framing models one for a public transportation infrastructure and other for a personal sensing device. In public transportation a Mobile Sensing Box (MSB) was installed in buses which contained the microcontroller, sensors for measuring Particulate Matter ($PM$) AND Carbon Monoxide ($CO$), a GPS module and a cellular modem. The module was powered through bus batteries and the collected data were transferred to server and visualized using google fusion tables interfaces. In the second framing, the system was installed in a car and connected over a bluetooth to a smartphone. This gave the user aware about the air quality while driving through a specific area.

The wireless sensor deployment in bus service \cite{Saha2017} proposed the idea of placing the sensor nodes in public transport buses in order to obtain real time data of pollutants. The WSN here is divided into sensor node and sink node in which the former collects the data from environment and latter aggregates the data from the sensor node and transmits to a server via a long range radio bands. These sink nodes are either installed in a T-junctions or an X-junction where most of the buses cross.
Another paper that  discusses an effective method to acquire air quality data in urban area is \cite {Shirai2016} in which the proposed system has the sensing unit kept on a public vehicle which keeps on moving in and around the city. The system has made use of garbage trucks and garbage patrol vehicles. The main focus is on pollutants like Particulate Matter ($PM$), Carbon Monoxide ($CO$) and Sulphur Dioxide ($SO_2$) and is also aided with GPS module. To remotely monitor the sensing conditions and to check for maintenance, a control center tool has also been developed. It consists of a map that tracks the route of the vehicles and the sensor data acquired by each vehicle. A monitor was developed along with the system to send the users the collected data. It estimates the amount of pollution inhaled by the user, by the acquiring the user's location from the mobile application and mapping it to the location-sensors value which has been already computed. It also considers the respiratory volume of the user to estimate the pollution inhalation.

A further method of transportation which was taken for understanding the air quality was public bicycle system. The bicycle borne sensor \cite{Xiang2016,Liu2015a} deploys an exhaust gas sensor and Particulate Matter ($PM$) sensor, GPS module and a microprocessor. The system collects data along with the location from the GPS module and is stored in micro SD card. The system is powered with the help of Lithium polymer batteries. When the subscriber returns the bicycle to dock station the data is transferred to data centre via bluetooth module and is then visualized using 'baidu' heat map. 

In \cite{Zhi2017} to understand air quality they used Unmanned Aerial Vehicle (UAV) as a medium to understand air quality.  In this the sensor board which measures six different pollutants that contribute to AQI index is mounted on the flight.  These flights are carried out in 30 days with 20 minutes of monitoring. Along with the sensor board, a smart phone is attached along which collects data from air sensor by establishing a bluetooth connection. The collected data is transferred to air quality analysis software that will display the real time monitoring value along with AQI value.

In a nutshell, the possibility of collecting redundant data is high in VSN as the chances of vehicle to get stuck in traffic congestion or for a car to be in a parking lot will collect excessive data that will reduce the data quality. The efforts to calculate AQI or AQHI is not taken as a priority except in \cite{Zhi2017} which uses UAV. The representation of  these indexes will make people aware about the air quality but most of the work in this category have failed to bridge the data gap.


\section{Wearable sensor network}


The individual effect of pollutants on health depends on the extent to which a person is exposed to the polluted environment. The understanding of the health effects could be achieved by observing the exposure-response relationship \cite{Dons2017} this will give an idea about the amount of pollutant inhaled by an individual. This could be achieved by using wearable sensor which helps to understand the effects of individual impact when exposed to polluted air. Research done in this section mainly focuses on improving the understanding of the personal health and exposure to air pollution \cite{Hu2015}. One such development is Mypart \cite{Tian2016}  from the University of California, is a wrist worn particle sensor that measures particulate matter of 10 microns or less.  The design of MyPart is based on traditional laser based photo diode system with improvement in airflow to remove light leakage, integration of structural design and circuitry for ambient visualization, BLE transceiver for low power networking and also mobile application for visualization. Two main issues tackled by MyPart is accuracy and calibration of sensor which no existing consumer sensor has addressed. A mobile application was also developed in addition to the hardware in order to demonstrate the volume of pollutant.

Another related work is 'Eco-mini'\cite{Fletcher2015} which is a wearable stand alone device for clinical use that measures Volatile Organic compounds (VOCs), sound level, air quality, temperature and humidity values. This system is based on a low power microcontroller (Atmel Xmega 128k) and consists of a GPS module for position identifying and a bluetooth module for data transfer. They modified the web server which was developed on a simple java script application. Another wearable work is 'citisense' \cite{Zappi2012} which is a system attached to a bag stripe which measures air pollutants like Nitrogen Dioxide ($NO_2$), Carbon Monoxide ($CO$) and Ozone  ($O_3$) along with environmental parameters such as temperature, humidity and barometric pressure. The collected data from the sensor is processed by the microcontroller (ATMEGA1284p) and transfers the data using a bluetooth module to a smartphone which does the data storing, analysis and data aggregation. The collected data is then transferred to back end to a web server where the user can get a personalized view of their data. They also developed a citisense android application that runs on the smartphone.
 
In \cite{Kim2010}, the research group in US developed an expressive T-shirt called 'WearAir' which indicates the measured volatile organic compound (VOC) through expressive patterns. The T-shirt is designed with a metaphor of a car emitting gases with four vertical arrays of LEDs which shows different frequencies depending on the concentration of VOC gas in the surrounding. When the wearer is exposed to dense VOCs the LED will blink rapidly. 
The authors of \cite{Hu2014} developed a novel system consisting of several sensors that gives a real-time feedback of an individuals exposure dose. This consist of arm sensors, chest sensor or even wrist sensors which measures various pollutant concentration ($CO$ in this case) and transfers to an android or ios application via bluetooth. They also calculates the inhaled dose of pollutant by calculating the volume of air inhaled into a person's lung per minute through an algorithm developed in \cite{Valli2013}. The inhaled dose of pollutant was calculated and compared  during various activities like jogging, bicycling and driving.

 There are also wearable sensor projects initiated in Vancover in association with University of British Columbia like TZOA \cite{tzoa} that can be clipped to the clothing and measures the air quality. It mainly measures $PM$ values and display in an application. These devices decreases the gap between individual and their understanding about the polluted air. In NewYork a striking project named 'Aircasting'\cite{aircasting} which provides the health and environment data using android 'Aircasting app'. The 'Aircasting'\cite{Han2010} platform includes a palm-sized air quality monitor which measures $PM_{2.5}$, relative humidity and temperature. The outside air is drawn through a sensing chamber and the particles are measured through light scattering method. It also includes a LED wearable apparel named 'Air casting luminisence'\cite{Luminescence} that illuminate LEDS according to the obtained sesnor measurement; variying from red for high intensity, then orange, then yellow and finally green for low intensity. 
 
 The emergence of such wearables make individuals to be more cautious about their health and encouraged them to stay fit. At the same time the public are not aware about these devices and do not prefer wearing T-shirts or carry a device for understanding the impact of pollution. Another issue to be focused is about the cost and stability of  connection due to which the measured data values wont be able to visualize.



\section{Community sensor sensor}

The development of portable sensor devices have paved way for a novel paradigm for monitoring the pollution known as crowdsourcing or participatory sensing. This gives an opportunity for any citizen to collect data and transfer it to a common platform like a webinterface. The collected data  from the participants gives a spatiotemporal view of the effect of pollution \cite{Kanhere2013}. In Sydney a low cost participatory system is deployed named 'Haze-watch' \cite{Sivaraman2013} for monitoring pollution in urban areas. In this, mobile sensor units was attached to vehicles and collected data were transferred using bluetooth to a mobile application which tags its location with date and time information. This data is then send to a cloud-based server which stores data and applies interpolation models \cite{Liao2006} to generate spatio-temporal estimates. Then using a web application the geo-referenced data is depicted as a contour map. 
\par
Intel has developed a prototype named 'Common Sense' \cite{Dutta2009} which is based on mobile participatory sensing \cite{Abdelzaher2007} that enables citizens to collect relevant data and involves in the decison making process. The system includes a handheld device that measures a couple of pollutants and uploads the value for visualization over the web using bluetooth or GPRS radios. This work was further tested by deploying it on muncipal fleet of street sweepers in the city of San Francisco \cite{Aoki2008}. Another community-driven sensing developed is 'OpenSense' \cite{Aberer2010} which focuses towards the utility of data by giving an idea about how the data collected from sensors needs to consumed. They have provided two use-cases first one is smart healthcare which by giving alerts on identifying the pollution induced diseases (like asthma, particle allergies, etc.) and next is urban planning by identifying polluted areas and identifying alternative routes. The system is deployed on mobile vehicles and stationary stations in Swizerland and the collected data is pipelined to a Global Sensor Network (GSN) from where the streamed data is processed and represented. In \cite{Hasenfratz2012} an outdoor participatory monitor was introduced called 'GasMobile' by connecting low-cost ozone sensor to an android smartphone. The collected data from the sensor is transfered to the phone and from which it is visualized using application as well as webserver. They have also implemented calibration procedures to the low cost sensors and the work claims to have high accuracy when compared to static measurement. The above mentioned research work in 'OpenSense' and 'GasMobile' have made opening for a further participatory sensing research in Swizerland supported by samsung called 'Exposuresense' \cite{Predic2013} that monitors user activities like walking, running, etc., from smartphones and understanding their exposure from obtaining datas from the already installed 'OpenSense' and 'GasMobile'. Their main idea here is to make use of the available smartphone for next generation healthcare.
\par
The growth of Micro-Electro-Mechanical System (MEMS) and Wireless Sensor Network (WSN) have made difference in the way how data is collected and understood from the physical world. 'G-Sense' \cite{Perez2010}, for Gloabl-Sense, is a work initated from the University of Florida in which they combine features of sensing platform applications like Location-Based Services (LBS) for tracking and location identification, Participatory Sensing (PS) for determining pollution index and other environment data, and Human-Centric Sensing (HCS) for heath related datas for specific group of users. The sensors collects the data and sents to a first-level integrator where all the data gets collected and from there using a data transport network it is transferred to the server that stores and performs data processing. It is from the server where the data visualization takes place which reports the data. Later there was another work which is considered to be the subset of G-Sense and named as 'P-Sense' \cite{Mendez2011} or Pollution-Sense. The architecture of this system is based on G-Sense in which external sensors are integrated into an Arduino development board. In this the data collection is based on Participatory Sensing (PS) and the goal here is to provide government official, doctors, and community developers with data so as to get a deeper understanding. They have also pointed out the research oriented challenges that needs to be addressed when building a community networked system like security, privacy, data visualization and working towards to achieve them. 

The work discussed in this category seems to be more promising but at the same time the quality of data obtained, getting public involvement for data collection and privacy issues \cite{Yi2015} are a few challenges researchers are trying to work towards it. The cost of maintenance in such community network in case of any damage is a crucial factor.



\section{Static sensor network}

 In this, the system is kept at a fixed location like traffic lights, street lights or any planned areas \cite{Pavani2017} which collects the pollutant values and transfers it to a visualization platform where the users can view it instantly. These systems are inexpensive and can be easily replicated or replaced. The system can be used for measuring either indoor or outdoor pollutants. Pollution in urban areas is increasing rapidly and due to which the number of people suffering chronic illness, permanent disability or even death are also increasing \cite{Wong2014}. The already existing station based environmental monitoring system are complex and costly hence there is a need to develop portable and low cost environmental monitoring system. The Integrated Environmental Monitoring System (IEMS)\cite{Wong2014} integrates different environmental detection sensors in a single  system and data from this system is used for processing and visualization. IEMS consist of Integrated Environmental Monitoring Device (IEMD) which consists of Microcontroller units, sensors, wireless communication module. They developed Handheld Remote Control Panel (RCP) for the system which is an android application that acts as an interface for the device control and handles the data exchange between IEMD and Web Server. Finally the web server that provides a real time data visualization and data analysis. These systems were placed on bus stops, bridges and even in the construction sites. In mauritius a research team developed a Wireless Sensor Network Air Pollution Monitoring System (WAPMS) \cite{K.Khedo2010} that designed a data regression algorithm named Recursive Converging Quartiles (RCQ) to remove duplicate data and then calculate AQI values. The array of sensor node collects the pollutant data and transfers it to cluster heads where the RCQ is applied to improve the efficiency and alleviate congestion problem. From the cluster head the data is send to server and represented using line graphs for each areas.

 Another related work is 'AirSense'\cite{Fang2016} which is an excellent approach to asses the indoor air quality. The author tries to introduce the idea of indoor air quality to the society by proposing a system which measures indoor pollutants. The system works through electronic sensors which are coupled to an Arduino (processing unit). The system not only extracts the data, but also provides its users with very effective visualization and analysis of the data. The researchers have done an excellent work on developing this robust system that would sense the pollution and provide education and awareness among the users. This system has made use of some machine learning algorithms to predict the pollution sources and forecast their behavior so as to provide intelligence to the system. 
The system has also got a smart phone application which gives the users a very effective interface for visualizations and understanding the data. In \cite{Liu2017} a different group of researchers from China developed the system 'Air-Sense' to monitor and predict the quality of air based on ZigBee network. The system uses 4 different type of sensor, respectievly are humidity, temperature, $PM_{2.5}$, Total volatile Organic Compund (TVOC includes the general organic gases) and a ZigBee transreciever for communication with network nodes. The prototype is tested for different areas in the house. It collects data in real time and using Bayesian mathematical statistics it predicts how accurate is the collected value to standard value predicted by WHO. 
\par

A static work which focuses on indoor air quality is demonstarted in \cite{Firdhous2017} in which the main focus is to understand the pollution in office environment where the pollution is triggered from the electronic devices and machines. In this the primary pollutant measured is ozone which is mainly emmited from a photocopier machines. The system is designed into different nodes where the sensing node contains the Arduino which process which collects the data from the ozone sensor. The measured data is transferred through a bluetooth link to a gateway node from where the data is formated as IP packets and forwarded over the Ethernet network to processing node. The data is saved in database and using 2D graph the concentration gets visualized. A research group from Harvard university developed a wireless networking testbed called as 'CitySense' \cite{Murty2008} in which multiple environmnetal sensors are attached to street lights. These sensors were deployed in Cambridge and data was uploaded to server using mesh networking like RoofNet \cite{Bicket2005}, TFA \cite{Camp2006} and CUWin \cite{cuwin2006}. Using a Web-based interface the data can be pulled from the server and made available to end users. The main feature of this research work is that the sensor nodes are powered from street lights and there is no constraint about long battery life.
\par
Liu J.H. et al \cite{Liu2011} developed a micro-scaled air quality monitoring system for understanding the $CO$ emmision from vehicles by integrating the sensor nodes with a gateway. The data collected from sensor is transferred to gateway using ZigBee communication link and from here meteorogical data and collected sensor data is forwarded to a central system through short message service via GSM. This centralised control system is supervised by the LabVIEW \cite{INSTRUMENTS2013} programming which helps in storing the data into MySQL database. They deployed the system on the main roads of Taipei city and obtained accurate values of pollutant concentration.

Our research work falls under static sensor network in which we have tried to integrate a system that measures all the major pollutant in the city of Prince George and providing AQHI and AQI values. Unlike the other system mentioned in the literature our main focus is to give a user specific data for better understanding of pollution.

 \section{Summary}

 In this chapter the research done in sensor network for understanding the air quality is categorized into four. We went through the system which is attched to a vehicle for understanding the pollutant concentration and gets categorized as vehicular sensor network. This category provides great mobility but at the same time accumalation of redundant data is high.
 Next category we mentioned is the sensors which could be worn or attached to a person. This gives a better understanding of individual health effects of pollutant and also the amount of pollutant inhaled. The work under this category has not gained a lot of attention as it demands the individual to carry the device. 
Participatory sensing is the next category in which the citizens performs the collection of data and it gets transferred to a common platform. Although the work done in this is more promising it faces challenges like privacy, data quality and maintenance cost. The final category is in which the system is placed at a planned area and called as static sensor network. Our research work falls under this section and we focus not just on collecting pollutant value but also making an effective visualization to reduce the data gap for users.