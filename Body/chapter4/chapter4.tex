\documentclass[12pt,a4paper,oneside]{report}

\topmargin -.5in
\evensidemargin 0in
\oddsidemargin 0in
\textheight 9in
\textwidth 6.5in
\usepackage{url}
\usepackage{graphicx}
\usepackage{amsmath,amssymb,amsfonts}
\newtheorem{theorem}{Theorem}
\usepackage{float}
\usepackage{mathtools}
\usepackage{tabularx}

\begin{document}

\section*{Calibration Of Low Cost Sensor}

With the development of sensor technology for air pollution it has attracted a majority of researchers as well as common people to explore and understand more about the pollutants and its effects. This has given freedom for one to set up their own monitoring system at residences, office or even at schools. The problem with this is to identify how accurate the data collected from these commodity sensors to the reference monitoring system. If the system is giving values which is way too different from the reference value then it brings down the advantages of this technology. This issue can be dealt through calibration which will reduce the uncertainty in data and makes the output more accurate. 
\par 
Calibration can be defined as the act of evaluating and adjusting the precision and accuracy of measurement equipment \cite{Kejuruteraan2018}. If the measured output from the sensor is not equals to the actual output then it shows that there is a need for calibration. Usually all the electronic instruments are calibrated according to a particular conditions in the laboratory and acquires a certification of calibration before its sold out in the markets. Even then the measured value does not reach accuracy as the condition or the environment where it was calibrated changes that leaves the user with some raw values that gives no information. This issue was taken up and explored by a few researchers in the US Environmental Protection Agency(EPA) and came up with three so called 'Straw-Man Approach' to improve the usability of such data and presented in the Air sensor Workshop \cite{Williams2013}. 

The first approach was by a signal-based calibration technique which requires the data from the remote stations, which is the reference station, to be broadcasted to the local station where the sensor is located and will receive this data and performs a single point calibration of the response. This approach would have been easy if the sensor was already equipped with the data collection and would process automatic calibration. 

The next option for calibrating is called the direct sensor calibration that involves placing the sensor in a chamber in which a known concentration of pollutant is set and response is observed. As the concentration of the pollutant is already known the output curve can be compared with it and calibrated accordingly. This is the most common method used for calibration as is often called as laboratory evaluation. Another way of approaching direct calibration is by inspecting the pre-defined response given by the manufacturer and checking how accurate the sensor is to the given concentration. In either case the calibration requires equipment and skills to give accurate concentration value.

The last option is by secondary data normalization in which the concentration values of the pollutant from the low cost sensor is normalized in accordance with the federal reference method (FRM) or federal equivalent method (FEM) analyzers. This approach is cost effective when compared to the other techniques and is less complex. This could be achieved by the use of a linear mathematical equation model that will convert the non-calibrated data into data of an acceptable form. The linear relationship equation $ y = mx + c $ where $ m $ is the slope and $ c $ is the intercept of the sensor raw data is compared with analyzer data and a relationship pattern is found out from this. The drawback of this is that the sensor data does not always give a linear response and thus not applicable for all the curves. 

Even though these 'Straw-Man Approach' was defined well it was still hard to implement it practically. This eventually led to the development of the 'Air Sensor Toolbox' by the Environmental Protection Agency (EPA) \cite{Williams2014} which introduced a tool called 'Macro Analysis Tool' (MAT) for performing comparisons of air sensor data with reference data and interpreting the results \cite{airsensorguidebook}.

\section*{Calibration Procedure}

The Air Sensor Toolbox provide guidelines for researchers, citizen scientists and developers with working of low cost sensor and its calibration.


\bibliographystyle{plain}
\bibliography{chapter4}
\end{document}