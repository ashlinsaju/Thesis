\documentclass[11pt]{article}

\topmargin -.5in
\evensidemargin 0in
\oddsidemargin 0in
\textheight 9in
\textwidth 6.5in
\usepackage{url}
\usepackage{graphicx}
\usepackage{amsmath,amssymb,amsfonts}
\newtheorem{theorem}{Theorem}
\usepackage{float}
\usepackage{mathtools}
\usepackage{tabularx}

\begin{document}

\section*{Calibration Of Low Cost Sensor}

With the development of sensor technology it has attracted a majority of researchers as well as common people to explore and understand more about the pollutants and its effects. This has given freedom for one to set up their own monitoring system at residences, office or even at schools. The problem with this is to identify how accurate the data collected from these affordable sensors to the reference monitoring system. If the system is giving values which is way too different from the reference value then it drops down the advantages of this technology. This can be done through calibration which will reduce the uncertainty in data and makes the output more accurate. Calibration is the process of evaluating and adjusting the precision and accuracy of measurement equipment \cite{Kejuruteraan2018}.


\end{document}