\documentclass[11pt]{article}

\topmargin -.5in
\evensidemargin 0in
\oddsidemargin 0in
\textheight 9in
\textwidth 6.5in


\begin{document}


\section{Development of Personal Integrated Environmental Monitoring System}\cite{Wong2014}

Pollution in urban areas are increasing rapidly and due to which the number of people suffering chronic illness, permanent disability or even death are also increasing. The already existing station based environmental monitoring system are complex and costly hence there is a need to develop portable and low cost environmental monitoring system. The mobile environmental sensing system integrates different environmental detection sensors in one system and data from this system can further be used for processing and visualization. The system which the paper suggested is Integrated Environmental Monitoring System(IEMS). IEMS consist of 3 components.
\begin{enumerate}
\item	Integrated Environmental Monitoring Device(IEMD)
\item	Handheld Remote Control Panel(RCP)
\item	Web Server.
\end{enumerate}
IEMD is portable device which consists of MCU, sensors, wireless communication module. The device is powered using six AA batteries. Sensors used are temperature and humidity, UV, PM, Noise Sensor.
The processor or MCU used in this is Arduino Nano board, and all the sensors are integrated into this MCU and the communication module used here is HC-06 blue tooth module.
RCP is an android application which is an interface for the device control. This does the data exchange between IEMD and Web Server. RCP on receiving information from IEMD, then it will transfer whole data to webserver using Apache HTTP.
Web Server provides a real time data visualization and data analysis. The paper has conducted a field test at several locations for evaluating the functionality of system. It showed that the system worked well. There were two main problems which cited:
\begin{enumerate}
\item GPS positioning accuracy was relatively low
\item Low battery life
\end{enumerate}



\section{Air Sense: An intelligent home-based sensing system for indoor air quality analytics}\cite{Fang2016}

Air sense is an excellent approach to assess the quality of indoor air. The paper tries to introduce the idea of Indoor Air Quality to the society by proposing a system which measures indoor pollution as the current society is miss-informed or ignorant about it. The system works through electronic sensors which are coupled to a Arduino (processing unit). The system not only extracts the data, but also provides its customers with very effective visualization and analysis of the data. This system has made use of some machine algorithms for the data analysis so as to provide intelligence to the system. The researchers of the paper have done an excellent work on developing this robust system that would sense the pollution and provide education and awareness among the users.

The main challenge in the system was to present the extracted data in such a way that, anyone irrespective of their backgrounds could understand the data and importance of Indoor Air Quality. This posed the system three sub challenges which included Pollution Event Detection, Source Identification and IAQ Forecast. All these were to be found from three sensor readings. This paper was able to successfully use some state of the art machine learning algorithms to accurately predict the pollution sources and forecast their behavior. The system has also got a smart phone application which gives the users a very effective interface for visualizations and understanding the data.
The main weakness of the paper is that the pollution sources have been limited to very few. The second weakness is that the prediction has been carried out on historical data rather than the data related to pollution in a particular area or state.


\section{IOT Enabled Proactive Indoor Air Quality Monitoring System for Sustainable Health Management.}\cite{Firdhous2017}
This system measures the indoor air quality. All though the paper discussed about a variety of pollutants and pollution events, the author's interest in office environment wherein the pollution to the gases from electronic devices and machines are high. Hence the system considers Ozone gas to be the primary pollutant. The author has put forward an idea of transmitting data through a Bluetooth channel and process the data in a Wi-Fi Network. The paper also discusses about the new construction design wherein the ventilation is very less which might lead into the presence of numerous sources of synthetic chemicals which has resulted in elevated concentrations of volatile organic compound inside the rooms. One such machine is the photocopiers, which is known to emit poisonous gases like ozone. The system consists of 4 nodes.
Sensing node which is a Arduino which acquires the data from the sensors
Ozone sensor which measures the level of ozone and sent it to the sensing node. 
Gateway node which used to transmit the acquired data through Bluetooth and finally a processing node which process the data.
Overall, the paper only discusses ozone as pollutant in the system neglecting all the other pollutants.

\section{Toward On-Demand Urban Air Quality Monitoring using Public Vehicles.}\cite{Shirai2016}

The paper discusses an effective method to acquire air quality data in an urban area. In the earlier systems the sensing module was fixed at different stations and therefore will only measure the air quality in the vicinity. The system proposed has the sensing unit kept on a public vehicle which keeps on moving in and around the city. The system has made use of garbage trucks and garbage patrol vehicles due to above mentioned reasons. The paper's main focus is in pollutants like PM, CO and SO. The system is also aided with GPS system so that the sensors values could be mapped onto the locations. To remotely monitor the sensing conditions and to check for maintenance, a control center tool has also been developed. It consists of a map, graph which tracks the route of the vehicles and the sensor data acquired by each vehicle. The next part of the paper is a monitor which was developed along with the system to send the users the collected data. It estimates the amount of pollution inhaled by the user, by the acquiring the user's location from the mobile application and mapping it to the location-sensors value which has been already computed. It also considers the respiratory volume of the user to estimate the pollution inhalation.

The paper proposes an effective and low cost system to understand the air quality in an urban area.

\section{Towards smart city: Sensing Air quality in city based on opportunistic crowd sensing.} \cite{Dutta2017}

A crowd sensing based air quality monitoring system is designed which aims at collecting and aggregating sensor data for air pollution monitoring in city. An air quality monitoring system(AQMD) was made and connected with smart phones and the information was gathered and shared to cloud. Air sense developed here is a 4 tier architecture consist of crowd which provides data, the dealing with data which means air quality sensing, transferring of data (data forwarding from smart phone to cloud) and finally the data analysis which is storage, aggregation and analysis. Finally, the result computed in tier 4 is sent back to tier 1 through tier 3. The AQMD consists of sensors, a Bluetooth module, Arduino and a power supply. Arduino interfaces between sensor and Bluetooth module. whenever there is a change in air quality Arduino communicates with Bluetooth module and then data are transmitted to smartphones via Bluetooth. In smartphone the data from Arduino is collected, transferred to cloud and provide AQI map.

\section{MyPart: Personal, portable, accurate airborne particle counting}\cite{Tian2016}

The primary air pollutant for major health issues is small airborne particles of size less than 10 microns and due to which a particle sensor named MyPart was developed which is of low cost. Mypart sensor differs from the other sensor available in market in the following features accessibility, flexibility, portability and accuracy. The paper compares Mypart to the existing sensor system such as thermal based gas sensors and states that either some system consists of low cost gas sensors which does a poor job of accurately measuring the actual gas concentration due to sensor response selectivity and gas interaction problems with sensor. Next comparison is done with low cost led and photodiode based sensors and states that these sensors give unreliable readings especially at lower concentration. Another comparison is done with laser and photodiode based sensors and points that these system performs poor in outdoor and there is ambient light leakage. The design of MyPart is based on traditional laser based photodiode system with improvement in airflow to remove light leakage, integration of structural design and circuitry for ambient visualization, BLE transreciever for low power networking and also mobile application for visualization. Two main issues tackled by MyPart is accuracy and calibration of sensor which no existing consumer sensor has addressed. A mobile application was also developed in addition to the hardware.

\section{Air sense: Indoor environment monitoring evaluation system based on ZigBee network} \cite{Liu2017}

In this paper a system was developed for indoor environment to monitor and predict the quality of air based on ZigBee network. The system is called air sense and it informs the user about the current indoor environment. The main two challenges faced by IOT systems said in the paper are first, there is a battery limitations for sensor nodes since each sensor work with a limited battery power and second the communication bandwidth of sensor is limited which means they can transmit information only for a limited distance. The air sense uses 4 different type of sensor viz: humidity, temperature, PM 2.5, TVOC (includes the general organic gases) and a ZigBee transreciever, battery. The system collects data in real time and using mathematical statistics(Bayesian) the prediction of environment is done. The collected data is compared with the already set standards predicted by WHO using Bayesian mathematical statistics.

\section{Characteristics and applications of small, portable gaseous air pollution monitors.} \cite{McKercher2017}
In this paper, there was a study conducted on each mobile monitoring system that are already available in society. They compared seven small portable air pollution monitors: The personal ozone monitors (2B Technologies Inc.), the SENSIT(Unitec), the cairclip(cairpol), the series 500 portable monitor (aeroqual Inc), the AGT environment sensor (AGT International), the smart citizen kit (acrobatic industries) and the air casting air monitor (habitat map). The selected monitors were on the basis of measuring either one or more of three specific gaseous air pollutants (O3, NO2, CO). These three gases are most common pollutants that is analyzed by scientist. Each of the seven system were compared on the basis of their precision, range, cost and sensor type. Sensor technologies used in each of these monitoring system are either metal oxide semi conduction(MOS), electrochemical(EC), non-dispersive infrared radiation absorption(NDIR) and photo ionization detection. Each of the four techniques are having its own individual characteristics, the common feature among them is that all of them can be employed in small, low cost air pollution monitors. Each of the monitoring system fits well in either personal exposure monitoring, supplementation of existing networks and citizen science. The main limitation of these monitoring system is about accuracy.

\section{Air Quality Monitoring Platform Based On Remote Unmanned Aerial Vehicle With Wireless Communication.}\cite{Zhi2017}

In this paper, they introduced a mobile and portable monitoring system based on UAV is developed. Current ways of monitoring includes satellites, ground monitoring stations, mobile monitoring, out of which ground stations are most common. These ground stations has its own limits on collection of air quality data because of the main reason that there locations are fixed. So in order to monitor the air quality around different region, the mobile monitoring was brought up. In this the monitoring platform is mounted onto the UAV with communication link to the ground, where the air quality analysis software is constructed to analyze the obtained data. The UAV used here is a hexa-rotor since it has redundant rotor wings and good performance such as wind resistance when compared to quadcopter.
The sensor board involved measuring six kinds of component: PM2.5, PM10, SO2, NO2, O3, CO. the sensor board in mounted on top of UAV and they are calibrated before takeoff. In order to communicate a smart phone is also mounted to monitoring box. The collected data about air quality are transmitted to ground station and analyzed using a software.  

\section{Sensor Deployment for air pollution monitoring using public transport system.}\cite{Yu2012}

Air pollution monitoring system using sensor network is mainly used for pollution monitoring. These sensor system basically collect data and analyze data.in order to increase the coverage with minimum number of sensors, these sensor network must be attached to some moving object and this is the main idea of paper. In this the sensor network is attached to a bus. The main area they work is the selection of bus routes in which to be monitored. For this they consider the whole coverage area to be divide into identical square grids and the bus routes are divided into segments. So the selection of bus route is formulated as bus sensor deployment problem(BSDP) and solving this is done by CRO technique.


\bibliographystyle{plain}
\bibliography{chapter2}
\end{document}