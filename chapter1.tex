\chapter{Introduction}
\pagenumbering{arabic}
\setcounter{page}{1}


Air pollution is stated as a complex mixture of gases and particles whose sources and composition vary over
space and time \cite{HealthEffectsInstitute2017}. The boom in the development of industries and technology have created an alarming situation - degrading the quality of air. It is a matter of serious concern and society is less aware of the impact that it causes to human health as well as to the surrounding. The World Health Organization (WHO) reported that 9 out of 10 people breathes polluted air which estimates a death rate of around 7 million every year \cite{who} \cite{WHO2010}. This has made many motivated individuals like researchers and communities to work towards in creating awareness among the people.

\section{Impact of Air Pollution}

Air pollution has significantly increased after the industrialization and urbanization have taken place. The burning of fossil fuels, exhaust from factories and industries, and mining operations are the major contributors to air pollution. The exposure to air pollutants causes premature deaths, cardiovascular disease, stroke, and other respiratory diseases. The state of global air 2017 has discussed the effects of long-term exposure to harmful air pollutants such as particulate matter which contributes to over 4 million premature deaths and is estimated to
double by 2050 if the issue remains unattended \cite{HealthEffectsInstitute2017}. Among the risk factors with the serious health issues, air pollution ranks the highest annually accounting for majority of deaths. The major impacts of air pollution are premature deaths, cardiovascular disease, stroke and other respiratory diseases.

\section{Background}

\section{Motivation}

One of the important components in solving this issue is to increase the awareness among all stakeholders, particularly common people about the current situation and its impact so that they can act on it. The conventional method of monitoring the air quality with the help of a few heavyweight expensive stationary monitoring systems typically installed by the state may not be effective enough for this task. To achieve the goal effectively and without further delay, pollution monitoring must become part of daily activity for everyone. For that the devices to monitor pollution must be small, portable, inexpensive, and part of a global system. With the technological advancement of low cost computing, communication, and sensing devices, and the revolution and the importance of open
source software \cite{Anthes2016}, we believe it is possible to build pervasive air pollution monitoring system with commodity hardware and open source software. Now the question is how to design such pollution monitoring devices faster and make them accessible to as many as possible.
\\
Achieving the above stated goal requires a suitable system framework that can help to accelerate the process of the design and implementation of a air pollution monitoring system using the of-the-shelf commodity hardware and open source software. There are some recent attempts in this direction, but none is comprehensive and simple enough to follow and build a air pollution monitoring system with a little or moderate effort. This paper is an attempt to fill that gap by first proposing a simple and comprehensive framework and then demonstrating its feasibility and use by creating our own pollution monitoring system that is operational in our lab. With some additional work, we are planning to release the framework with suitable documentation to the public. If accepted for a presentation in the conference, we plan to demonstrate a fully developed and tested pollution monitoring system with a proper 3D printed cases at the the conference.Our contribution is a step towards inspiring and motivating not only the public to use the device, but also many amateur electronic hobbyists to buy the hardware locally and download the associated software to build their own pollution monitoring device to aid the mission of creating a pervasive pollution monitoring system.


\section{Research Problem}
The aim is to propose, design and develop an air pollution monitoring system using off the shelf
hardware and open source software, with the following objectives in mind.
\begin{enumerate}
   
    \item To educate the common people on adverse effect of air pollution by showing how polluted
    the vicinity is.
    
    \item To influence the behavior of people by representing the concentration of pollutant as well
    as AQHI(Air Quality Health Index) which provides the seriousness that pollutants cause
    to health.
    
    \item To give an idea of how to integrate all the hardware components to a processor and also
    make an independent software, which can be accessed anywhere in the world.
    
    \item To encourage and help citizen science to solve the issue of air pollution and give more
    understanding to the impact it cause to human health and environment.

\end{enumerate}


\section{Thesis Contribution}

\section{Structure of the Thesis}