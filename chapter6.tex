\chapter{Conclusion and Future work}

The advancement in sensor technologies has given a lot of opportunities for researchers to explore different fields. These technical advancements have opened the door for solving major issues and one of them is environmental issues like air pollution. Air pollution has been a major concern ever since industrialization and modernization has hit the world and from then onwards it went on increasing. There are several works done under this category of research. In this chapter we will be summarizing the work done and also how this research project can be extended in future.
    
\section{Summary}

 In our work, we have attempted to build a low-cost sensor system that measures the pollutants specific to the City of Prince George. The system which we built is simple, easily replicable and all the sensors used for the study are low-cost and readily available in the market. We have implemented calibration procedure to ensure the quality of measurement obtained from the system. This calibration procedure was a replication of MAT tool developed by Environmental Protection Agency (EPA). The calibrated data was visualized in an IoT platform for three categories of users which are layman, data scientists, and policymakers. The system was deployed for six days for collecting the measurements and from the graphs, the measurements obtained from the system show correlation with the reference data obtained from Downtown Prince George.

\section{Challenges and Directions Identified}


Even though the system which we build showed a good response when compared to the reference system there are few downsides for the created system. 
One of the main drawbacks of the system was in the calibration procedure that we have applied. As discussed in chapter 4, we have replicated the MAT that uses linear regression algorithm for obtaining the calibration curve. From our observations, it can be seen that there are certain data points that are significantly different from the rest of the data points and these are called outliers. These outliers happen when there a variablility in the measured value or it could be a recording error from the sensor. In either cases we could visually plot the data to see the outliers in the measurements and understand from these points. %In our work, we have not focused on the outliers in the measured value and this could be a good addition to our work. Another focus in the calibration is the interval for recalibration. In our work we have only collected data for six days which makes it harder to determine this interval.

Next point of discussion is regarding the location of the sensor system for measurement. We have placed our sensor system in College Heights, Prince George that is around 9 km far from the reference system. Pollutant like Particulate Matter is very location-dependent and small activity could cause variation to the data. One solution for this issue could be possible field co-locations of the sensor to the reference system to understand on their performance \cite{austin2015laboratory}. This means that collecting the data from the same location to that of reference station will give more understanding. 

Another matter of concern is regarding the unknown duration that the applied calibration is valid or to find out the recalibration interval so that the sensor's performance would not affect. As the sensor used for our work are low-cost sensors the problems may arise like drift in sensor's performance or faulty of sensor at deployment. To avoid these issues there should be routine checks on sensor's performance and also should determine the recalibration interval for the sensor. These interval could be different for different sensors and should be updated periodically as per the procedures in Chapter 4, section 4.2.


One of the main goal of our work was to build an efficient system with the help of low-cost sensors. The problem with the low-cost system is that there are variety of manufactures who produce these sensors. In our case we referred the Air Sensor Guidebook discussed in Chapter 3, section 3.1 for the selection of sensors. Even then the data sheet given by the manufactures lacks in detail, accuracy and reproducibility is unknown for most of the sensors. Another difficulty with low-cost sensor is regarding their sensitivity to changing environmental conditions mainly temperature, humidity, interference from other sensors, and signal drift. We also have not much knowledge of their long term stability and performance of the sensors as we only have measurements for six days. 


Moving on to the data handling and data presentation section there are certain areas that we have not covered. The collected data is stored in ThingSpeak server which is not accessible on the world-wide data network.


In our case we are representing the real time data through the developed software. 





\section{Future Works}

There are many ways this work could go forward and some of them I have listed below:
\begin{enumerate}
    \item The system could be elaborated with different set of pollutants. In the present system we have only used a certain set of pollutant but there is a possibility to expand the system by adding another pollutant like TRS (Total Reduced Sulphur) or $SO_{2}$.
    
    
    \item The sensors we have used for our studies will not work during extreme winter weather. There are sensors available in market which can replace the sensors used in our studies. 
     
    \item We could make the system better by applying collecting more data points which will give us a better understanding on the calibration procedure and the reliability of the sensors.
    
    \item The calibration used here is linear regression which is an estimate of the 'best line' through the data. There will be some uncertainty associated to this and hence we can explore other calibration techniques.
    
    \item Location where the instrument is placed is another main factor for measurement. By keeping the system as close as to the reference system will give a better idea on the pollutant data.
    
    \item We could work on the outlier data to understand if they have significant effect on the measurement. This could be done by filtering the outliers to understand the effects. 
    

\end{enumerate}



