\chapter{Conclusion and Future work}

The advancement in sensor technologies have given a lot of opportunities for researchers to explore in different fields. These technical advancement have opened the door for solving major issues and one of them is environmental issues like air pollution. Air pollution have been a major concern ever since industrialization and modernisation has hit the world and from then onwards it went on increasing. There has been immense number of research works been done in this area.

In this work, I have attempted to put my knowledge of electronics to build a sensor system which measures the pollutants specific to Prince George. The system which I have tried to build is very simple , easily replicable and all the sensors used are low cost and readily available. Then I have also put my hands on the visualization of the collected data and calibration. For visualization we have created an IoT platform in which the data was represented for three category of people.
The system was deployed and the data was observed and it could be seen that the data after calibration are close to the reference system. From the graphs it can be seen that the values obtained from the system showes a better correlation with the reference data. One of the major drawback with these low cost system is that they are highly vulnerable to damage by external factors like rain. Another drawback is that the system needs constant recalibration in order to provide accurate data.

\section{Future Works}

There are many ways this work could go forward and some of them I have listed below:
\begin{enumerate}
    \item The system could be elaborated with different set of pollutants. In the present system we have only used a certain set of pollutant but there is a possibility to expand the system by adding another pollutant like TRS (Total Reduced Sulphur) or $SO_{2}$
    \item The visualization code can be extended and added different features to it. The software could be made more user friendly.
    \item In this work the calibration equations are put into the system manually. This can be coded and automated so there will be less effort for the user to have it recalibrated.

\end{enumerate}



